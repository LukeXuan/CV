% adopted from
% https://raw.githubusercontent.com/xdanaux/moderncv/master/examples/template.tex
% thanks: Xavier Danaux (xdanaux@gmail.com).
% author: Yixuan Chen (xlk@umich.edu)


\documentclass[11pt,letterpaper,sans]{moderncv}
% moderncv themes
\moderncvstyle{classic}                             % style options are 'casual' (default), 'classic', 'banking', 'oldstyle' and 'fancy'
\moderncvcolor{black}                               % color options 'black', 'blue' (default), 'burgundy', 'green', 'grey', 'orange', 'purple' and 'red'
%\renewcommand{\familydefault}{\sfdefault}         % to set the default font; use '\sfdefault' for the default sans serif font, '\rmdefault' for the default roman one, or any tex font name
%\nopagenumbers{}                                  % uncomment to suppress automatic page numbering for CVs longer than one page

% adjust the page margins
\usepackage[scale=0.75]{geometry}
%\setlength{\hintscolumnwidth}{3cm}                % if you want to change the width of the column with the dates
%\setlength{\makecvheadnamewidth}{10cm}            % for the 'classic' style, if you want to force the width allocated to your name and avoid line breaks. be careful though, the length is normally calculated to avoid any overlap with your personal info; use this at your own typographical risks...

% personal data
\name{Yixuan}{Chen}
\title{curriculum vitae}                             % optional, remove / comment the line if not wanted
% \address{street and number}{postcode city}{country}% optional, remove / comment the line if not wanted; the "postcode city" and "country" arguments can be omitted or provided empty
\phone[mobile]{+1~(734)~789~0357}                   % optional, remove / comment the line if not wanted; the optional "type" of the phone can be "mobile" (default), "fixed" or "fax"
% \phone[fixed]{+2~(345)~678~901}
% \phone[fax]{+3~(456)~789~012}
\email{xlk@umich.edu}                              % optional, remove / comment the line if not wanted
\homepage{blog.xlk.me}                         % optional, remove / comment the line if not wanted
\social[github]{LukeXuan}                              % optional, remove / comment the line if not wanted
% \quote{Some quote}                                 % optional, remove / comment the line if not wanted

\renewcommand*{\bibliographyitemlabel}{[\arabic{enumiv}]}
\begin{document}
\makecvtitle

\section{Education}
\cventry{2017--2019}{Bachelor}{Computer Science Engineering}{University of
  Michigan}{\textit{4.0}}{Dual Degree Program}
\cventry{2015--2019}{Bachelor}{Electrical and Computer Engineering}{Shanghai
  Jiaotong University}{\textit{3.6}}{}
\cventry{2012--2015}{High School}{}{Shanghai High School}{}{}

% \section{Master thesis}
% \cvitem{title}{\emph{Title}}
% \cvitem{supervisors}{Supervisors}
% \cvitem{description}{Short thesis abstract}

\section{Research Interests}
\cvitem{}{Formal verification, programming languages, operating systems and
  applying formal methods in large and concurrent software systems}
\section{Research}
\cventry{May. -- Aug. 2018}{Research intern}{prof. Andrew Appel}{Princeton
  University}{}{Designing and verifying high performance KV-database implemented
  in C
  \begin{itemize}
  \item Using Verified Software Toolchain and Verifiable C to prove
    the functional correctness of C programs
  \item Designing and verifying a high performance in-memory KV-database based on
    MassTree and SQLite cursor
  \item Exercising the modular verification approach and finding the correct
    abstractions
  \item Expecting results and methodologies to be present at POPL student
    research competition and other publications in 2019
  \end{itemize}}
\cventry{2017--2018}{Research assistant}{prof. Manos Kapritsos}{University of
  Michigan}{}{Verification tool on concurrent programs
  \begin{itemize}
  \item Understanding and modifying the Dafny language and program verifier
  \item Designing algorithms for automatic derivation of abstract syntax tree
    mappings to generate refinement proofs between transformation of programs
  \item Expecting results to be present at OSDI in 2019
  \end{itemize}}

\section{Academic Experience}
\cventry{2018 Fall}{Teaching assistant}{EECS 482 Operating Systems}{University
  of Michigan}{}{
  \begin{itemize}
  \item 250-student upper level technical elective course
  \item duties including holding office hours and lab teaching
  \end{itemize}
}
\cventry{2018 Jul.}{Student volunteer}{DSSS 2018}{Princeton}{NJ}{DeepSpec
  Summer School 2018
\begin{itemize}
  \item Assisting participants on Coq proofs with Verified Software Toolchain
  \end{itemize}}
\cventry{2018 Jan.}{Participant}{POPL 2018}{Los Angeles}{CA}{
  Principles of Programming Languages 2018}

\section{Publications}
\cvitem{2019}{\textit{Verification of a Cache-optimized Data Structure}, by
  Yixuan Chen, Aur\`ele Barri\`ere, Lennart Beringer, and Andrew W. Appel,
  accepted to appear in poster session at POPL 2019 Student Research Competition
  (POPL '19 SRC), Lisbon, Portugal}

\section{Industrial Experience}
\cventry{2016--2017}{Intern}{Apple Inc.}{Shanghai}{}{
  \begin{itemize}
  \item Developed concurrent software systems used for audit and version control
    of test stations
  \item Wrote detailed code documents and deployment instructions for the
    systems to be maintained after leaving
  \end{itemize}}

\section{Projects}
% 591 paxos, 591 research, 483 compiler, JOS operating system(osdev)
\cventry{2018 Winter}{Course Project}{Fault-tolerant distributed chat
  server}{}{EECS591 (Graduate Distributed Systems)}{
  Implemented chat server with replication that tolerates benign failures and
  unreliable channels
  \begin{itemize}
  \item Understood \textbf{Paxos} and \textbf{multi-Paxos} algorithms and
    implemented in Python
  \item Dealt with non-deterministic bugs introduced by unreliable channels
  \end{itemize}
}
\cventry{2018 Winter}{Course research project}{On investigating mixer’s impact on
  database performance}{EECS 591 (Graduate Distributed Systems)}{}{
  Identifying application-aware mixer's impact on the performance of database
  \begin{itemize}
  \item Benchmarked the database performance with TPC-W datasets
  \item Reasoned about the database performance in terms of network,
    concurrency, database and application logic 
  \end{itemize}
}
\cventry{2018 Winter}{Course project}{Compiler for COOL}{EECS 483 (Undergraduate
  Compilers)}{}{
  Implemented a compiler for COOL (subset of Java) programming language, written in OCaml and
  compiles COOL program into x86\_64 assembly program
  \begin{itemize}
  \item Implemented a full-stack from lexical analyzing to code generation
  \item Used intermediate language to achieve extensibility
  \item Generated assembly file that can be assembled into ELF file and
    directly executable on Linux
  \end{itemize}
}

\section{Programming Languages}
\cvitem{Functional}{OCaml, Coq}
\cvitem{Imperative}{C, C++, C\#, Pascal}
\cvitem{Scripting}{Python, Javascript, Shell (and variants)}
\cvitem{Others}{Dafny, LaTeX, HTML/CSS, Verilog, Matlab, Mathematica}

% \section{Languages}
% \cvitem{English}{Proficient, TOEFL: 110}
% \cvitem{Chinese}{Native speaker}

% \section{Interests}
% \cvitem{Guitar}{Novice}

\end{document}

%% end of file `template.tex'.